\documentclass[10pt,unicode]{beamer}
\usepackage{luatexja-fontspec}
\renewcommand{\kanjifamilydefault}{\gtdefault}
\usetheme{th}
\AtBeginSection{
	\begin{frame}
		\tableofcontents[currentsection]
	\end{frame}
}
\AtBeginSubsection{
	\begin{frame}
		\tableofcontents[currentsubsection]
	\end{frame}
}
\renewcommand{\baselinestretch}{1.2}
\title{T.H.先生の講義資料っぽいやつ}
\subtitle{応用コンピュータリテラシー 第100回}
\author{R02092\inst{1}}
\institute{\inst{1}高知工科大学 情報学群 学生}
\titlegraphic{\includegraphics[width=175px]{kut-000.png}}
\date{2024年5月26日}
\logo{\includegraphics[clip,width=20px]{kut-001.png}}
\begin{document}
	\begin{frame}[plain]
		\titlepage
	\end{frame}
	\begin{frame}{講義予定}
		ないです.
	\end{frame}
	\begin{frame}{本日の目標}
		\begin{enumerate}
			\item \LaTeX でT.H.先生の講義資料っぽいやつを作れるようになる.
			\item 演習課題の提出方法を習得する.
		\end{enumerate}
	\end{frame}
	\section{テキストの確認}
	\subsection{数学書の文書の読みかた}
	\begin{frame}
		\frametitle{数学書の文書の読みかた}
		\begin{block}{倍数}
			$a$を整数,$b$を$0$でない整数とする.$a=bq$を満たす整数$q$が存在するとき,
			$a$を$b$の倍数という.
		\end{block}
	\end{frame}
	\subsection{素数}
	\begin{frame}
		\frametitle{素数}
		\begin{exampleblock}{素数の例}
			\begin{itemize}
				\item $2$ は素数.
				\item $3$ も素数.
			\end{itemize}
		\end{exampleblock}
		\begin{alertblock}{素数ではない数}
			\begin{itemize}
				\item $4$ は素数ではない.
				\item $57$ は素数ではない.
			\end{itemize}
		\end{alertblock}
	\end{frame}
	\section{Processingクイックツアー}
	\subsection{情報科学応用の資料に出てきたやつ}
	\begin{frame}{C言語との相違点}
		\begin{itemize}
			\item 変数の使用前に変数の型を宣言
			\begin{itemize}
				\item forループの中でのみ利用する変数をforループの初期化部分で宣言可能
				(C++/Java)に由来
				\begin{itemize}
					\item 変数を宣言したforループ内だけで通用,ループの外側からの参照不可能
				\end{itemize}
				\item プログラム途中での変数宣言が可能
			\end{itemize}
			\item 実数を表す標準的な型がProcessingではfloat(Cではdouble)
		\end{itemize}
	\end{frame}
\end{document}